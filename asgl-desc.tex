\documentclass[runningheads,a4paper]{llncs}

\usepackage{amssymb}
\setcounter{tocdepth}{3}
\usepackage{graphicx}

\usepackage{url}
\newcommand{\keywords}[1]{\par\addvspace\baselineskip
  \noindent\keywordname\enspace\ignorespaces#1}

\begin{document}

\mainmatter  % start of an individual contribution

% first the title is needed
\title{ASGL}
\subtitle{Argumentation Semantics in Gecode and Lisp}

% a short form should be given in case it is too long for the running head
%\titlerunning{Score Manipulation in PWGL using KSQuant}

% the name(s) of the author(s) follow(s) next
%
% NB: Chinese authors should write their first names(s) in front of
% their surnames. This ensures that the names appear correctly in
% the running heads and the author index.
%
\author{Kilian Sprotte}

%
%\authorrunning{KSQuant - Complex Score Manipulation in PWGL through Quantization}
% (feature abused for this document to repeat the title also on left hand pages)

% the affiliations are given next; don't give your e-mail address
% unless you accept that it will be published
\institute{FernUniversit\"at in Hagen,\\
  Universit\"atsstra{\ss}e 47,\\
  58084 Hagen, Germany\\
  \email{kilian.sprotte@gmail.com}}

%
% NB: a more complex sample for affiliations and the mapping to the
% corresponding authors can be found in the file "llncs.dem"
% (search for the string "\mainmatter" where a contribution starts).
% "llncs.dem" accompanies the document class "llncs.cls".
%

%\toctitle{KSQuant - Complex Score Manipulation in PWGL through Quantization}
%\tocauthor{Score Manipulation in PWGL using KSQuant}
\maketitle

\begin{abstract}
  ASGL is a solver for argumentation semantics, capable of answering
  queries with respect to grounded, complete, preferred, and stable
  semantics. It is built based on GECODE, a generic CSP solver. ASGL
  itself is mainly written in Common Lisp. This paper presents a
  description of its system components and, for a selection of the
  computational tasks, provides details on how ASGL approaches them.
\end{abstract}

\section{Introduction}\label{sec:introduction}

ASGL~\cite{asgl} is a solver for argumentation semantics. Given an
argumentation framework $AF = (Ar, att)$, answers to queries with
respect to grounded, complete, preferred, and stable semantics can be
computed. Specifically, for each semantic, ASGL allows to enumerate
one or all extensions and to report on the status of an argument,
while taking either a credulous or a skeptical point of view.

ASGL is mainly written in Common Lisp and CLOS~\cite{Steele:1990:CLL},
with some parts written in C++, which is used in the area of low-level
parsing of the input file and, more importantly, to interface to
GECODE~\cite{gecode}, a generic CSP solver library, which is used in ASGL as a
backend.

The plan of the paper is as follows. Section~\ref{sec:components}
presents GECODE and ECL~\cite{ecl}, the Common Lisp implementation, on
which ASGL is built. Then, the realisation of the computational tasks
is exemplified with respect to grounded semantics in
Section~\ref{sec:grounded} and preferred semantics in
Section~\ref{sec:preferred}. Section~\ref{sec:reductions} details
reductions of queries performed by ASGL and
Section~\ref{sec:conclusion} gives a brief conclusion.

\section{System Components}\label{sec:components}

GECODE, the generic constraint development environment,
is a toolkit for developing constraint-based systems and
applications. It is an efficient, generic CSP solver, featuring among
others finite domain integer variables and finite set variables and
constraints (see Section~\ref{sec:preferred}). A C++ library that can
be easily integrated with other systems and which is also very open to
be extended by user code. For instance, it is possible to program new
search engines that can be regarded \textit{en par} to the already
built in ones, such as depth-first search and branch-and-bound search,
without requiring the user to dive into low-level code.

The abstraction that new search engines can be programmed with is
called computation space \cite{Engines:97}. A computation space, which
is a first-class citizen, encapsulates a speculative computation
involving constraints. New constraints can be posted on a
space. Another operation is to wait for a space to become stable,
i.e. propagation reaches a fixpoint (see Section~\ref{sec:grounded}).

ECL is an implementation of Common Lisp that features a
byte-code compiler, but can also compile to C code, which is then
further compiled by the hosts native compiler. This allows for easy
integration with C or C++ libraries. While bindings to such libraries
are usually generated with a tool such as SWIG~\cite{swig}, ECL has a
feature that makes this unnecessary. It allows the user by the use of
an inline construct \textit{ffi:c-inline} to directly emit fragments of C
or C++ code as part of a Lisp function defintion. These fragments are
placed (almost) verbatim within the generated C code.

As an example, this allows for the definition of a Lisp function
\textit{space-status} for the beforementioned operation on a space,
which calls the underlying GECODE method \textit{status}.

\begin{scriptsize}
\begin{verbatim}
(defun space-status (space)
  (let ((status
         (ffi:c-inline (space) (:pointer-void) :int
                       "{
// wait for space to become stable, then retrieve status
Gecode::SpaceStatus status = (((Gecode::Space*)(#0))->status());

switch (status) {
case Gecode::SS_FAILED: @(return 0) = 1; break;
case Gecode::SS_SOLVED: @(return 0) = 2; break;
case Gecode::SS_BRANCH: @(return 0) = 3; break;
default: @(return 0) = 100; break;
}
}")))
    (ecase status
      (1 :failed)
      (2 :solved)
      (3 :branch))))
\end{verbatim}
\end{scriptsize}

ASGL makes use of typical Lisp features, such as CLOS, the Common Lisp
Object System, macros (see Section~\ref{sec:reductions}) and
first-class functions to orchestrate on a high-level the parsing of
the command-line arguments and input file, the creation and
appropriately constraining of a computation space, the invocation of a
search engine and then, finally, the formatting of the output.

\section{Grounded Semantics}\label{sec:grounded}

Exactly one grounded extension exists. It contains all the arguments
which are not defeated, as well as that arguments which are directly
or indirectly defended by non-defeated arguments. An algorithm to
compute this extension in linear time is given by
Caminada~\cite{Modgil2009}; it progresses iteratively until a fixpoint
is reached.

In ASGL, no special purpose algorithm for grounded semantics has been
implemented. A computation space is created as for the other semantics
with an array of $|Ar|$~boolean variables: ASGL uses an
extension-based encoding for solutions. Constraints are then posted on
the space and the status is queried by \textit{space-status}, which
first waits for the space to become stable, i.e. propagation to reach
a fixpoint. Subsequently, the grounded extension can be read from the
space by including all arguments whose corresponding variables are
instantiated to \textit{true}.

\section{Preferred Semantics}\label{sec:preferred}

The preferred extensions are all those complete extensions that are
maximal with respect to set inclusion. In the general case, more than
one preferred extension exists.

The task of computing some preferred extension is implemented in ASGL
like a classical optimization problem with branch-and-bound search. As
soon as one solution has been found, all further solutions are
constrained to be better than the current solution. If no more
solutions can be found, the current solution is maximal.

In the case of preferred extensions, an extension is better than
another iff it is a proper superset of the other. In order to allow
this kind of constraint to be posted, ASGL makes use of an additional
set variable that represents the extension as a set. For consistency,
the set variable is connected to the array of boolean variables by a
channeling constraint. A branch-and-bound search engine is already
part of standard GECODE. This allows for a straightforward
implementation of this strategy. The search for a maximal solution is
further supported by a value heuristic: When a choice needs to be
made, an argument is considered to be included first, before it is
considered to be excluded.

No built-in search engine in GECODE can be used to efficiently
enumerate all preferred extensions. In the development of ASGL, an
attempt has been made to implement a \textit{multi-bab-engine} for
this purpose. This engine essentially keeps a master computation space
that stays unmodified by individual searches. Only a clone of the
master is passed to the built-in branch-and-bound search
engine. Whenever this engine finds a preferred extension, the master
is constrained not to be a subset of this extension and the process
repeats.

Unfortunately, this strategy turned out to be slower than filtering
all complete extensions for maximality -- at least for small input
graphs. In the current version of ASGL, this work has therefore been
abandoned in favor of this more simple approach. We expect that
repeatedly restarting from the master space incurred too much overhead
by repeating work already performed in previous invocations. This
effect could possibly be mitigated by utilizing no-good learning or by
employing more sophisticated variable ordering heuristics, such as
accumulated failure count or activity that build on
information gained from previous searches.

\section{Reductions}\label{sec:reductions}

ASGL allows the product of grounded, complete, preferred, and stable
semantics and enumeration of some or all extensions, credulous and
skeptical inference, in total 16 different problems to be solved. In
this problem space numerous reductions are possible. ASGL currently
makes use of the following rules, which are -- thanks to Lisp macros
-- written exactly as given here in the source code:

\begin{enumerate}
\item \texttt{(translate (:se :co) -> (:se :gr))}
\item \texttt{(translate (:ds :co) -> (:ds :gr))}
\item \texttt{(translate (:dc :pr) -> (:dc :co))}
\end{enumerate}

The first and second rule are quite simple: 1. When asked for some
complete extension, one could simply compute the grounded
extension. 2. a) If an argument is included in all complete extensions,
it is also included in the grounded extension. b) The grounded extension
is a subset of all complete extensions, therefore an argument included
in the grounded extension is included in all complete extensions.

3. The third rule is more subtle. It states that whether an argument
is included in some preferred extension can be reduced to the question
of whether the argument is included in some complete extension. This
can be shown like this: a) If an argument is included in some
preferred extension, it is also included in some complete
extension. b) If an argument is included in some complete extension,
this extension is either maximal -- a preferred extension or a
preferred extension must exist that is a superset of the complete
extension, hence it includes the argument.

\section{Conclusion}\label{sec:conclusion}

\bibliography{bibdb}{}
\bibliographystyle{splncs03}

\end{document}
